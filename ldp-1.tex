\documentclass{article}
\usepackage{amsmath}
\usepackage{amssymb}
\usepackage[thehwcnt=1]{iidef}
\DeclareMathOperator{\mF}{\mathcal{F}}
\usepackage[utf8]{inputenc}
\thecourseinstitute{Tsinghua-Berkeley Shenzhen Institute}
%\title{ldp-1}
%\author{zhaof17 }
%\date{March 2021}
\thecoursename{Large Deviation Theory}
\theterm{Spring 2021}
\begin{document}

\courseheader
\name{Feng Zhao}

\begin{enumerate}
\item
\begin{enumerate}
    \item Suppose $A_i = \bigcup_{j \in N} \{a_{ij}\}$, to show
    $\bigcup_{i \in N} A_i$ is countable. We construct a mapping
    $(i,j) \to \frac{(i+j)(i+j+1)}{2} + j$, which is a 1-1 mapping
    from $N^2$ to $N$. Therefore, $\bigcup_{i \in N} A_i$ is countable.
    \item $Q$ can be regarded as the subset of $N^2$. By the property
    that "Any subset of a countable set is countable", we conclude that
    $Q$ is countable.
\end{enumerate}
\item
\begin{enumerate}
    \item We can verify $\emptyset  \in \mF_0$. For
    $A=\cup_{i=1}^n [a_i, b_i)$, $A^c = [0, a_1) \bigcup \cup_{i=1}^{n-1} [b_i, a_{i+1}) \bigcup [b_n, 1) \in \Omega$. Finally, for $A,B \in \mF_0$
    we can decompose it into $A=A_1\cup C, B=B_1 \cup C$ where $C=A\cap B, A_1=A\backslash C, B_1=B\backslash C, \in \mF_0$. We can verify that
    both $A_1, B_1, C$ are finite unions of intervals (left-close, right-open), and $A_1, B_1, C$ are disjoint.
    Then $A\cup B=A_1\cup B_1 \cup C$ is the collection of interval representations, also finite unions of intervals.
    $A, B \in \mF \Rightarrow A\cup B \in \mF_0$. Therefore, $\mF_0$
    is an algebra. Consider $A_i = [\frac{1}{n}, 1)$, then
    $\cup_{i=2}^{+\infty} A_i = (0, 1) \not\in \mF_0$. Therefore,
    $\mF_0$ is not a $\sigma$-algebra.
    \item Let $A=\cup_{i=1}^n [a_i, b_i)$, $B=\cup_{i=1}^m [c_i, d_i)$,
    $b_n=d_m=1$ is impossible since $A \cap B= \emptyset$.
    If one of $b_n, d_m=1$, $P(A\cup B)=1$ and $P(A)+P(B)=1$.
    If $b_n, d_m < 1$, $P(A\cup B)=0$ and $P(A) + P(B)=0$. In either
    case, $P(A\cup B)=P(A)+P(B)$. The finite additivity is satisfied.
    \item Consider $A_n = [1-\frac{1}{n}, 1-\frac{1}{n+1})$ for $n\geq 2$, $A_n, n=2,3,\dots$ are disjoint, whose union is $[\frac{1}{2}, 1)$.
    But $P([\frac{1}{2}, 1)) = 1$ while $P(A_n)=0$. The property of countable additivity is not satisfied.
    \end{enumerate} 
    \item Suppose $\sum_{n=1}^{\infty} P[X_n>c]$ converges, then
    for any $\epsilon$, there exists $N$ such that
    $\sum_{n=N}^{\infty} P[X_n>c] < \epsilon$.
    By union bound,
    $P[\sup_n X_n = \infty] \leq P[\exists n > N, X_n > c] \leq 
    \sum_{n=N}^{\infty} P[X_n>c] < \epsilon$.
    Since $\epsilon$ can be arbitrarily small, $P[\sup_n X_n = \infty]=0$.
    
    For the other side, we will show that
    $\forall c > 0, \sum_{n=1}^{\infty} P[X_n > c] = \infty \Rightarrow P[\sup_n X_n = \infty]=1$. Let $A_n = \{w|X_n(w) > c \}$,
    then $A_n$ are independent, 
    by Borel-Contelli Lemma, we have $P[\cap_{n=1}^{\infty} \cup_{i=n}^{\infty} \{w|X_n(w)>c\}]=1$. Therefore,
    $P[ \cup_{i=n}^{\infty} \{w|X_n(w)>c\}]=1$ holds for any $c,n$.
    Then $P[ \cap_{i=n}^{\infty} \{w|X_n(w)\leq c\}]=0 \Rightarrow
    P[ \{w|\sup_n X_n(w)\leq c\}]=0$. By union bound,
    $P[\cup_{c=1}^{\infty}\{w | \sup_n X_n(w) \leq c \}]=0$, then
    taking the complement we have
    $P[\{\sup_n X_n = \infty\}]=1$.
    
   \item For any $c \in \mathbb{R}$, $\{w| \max\{X_1(w), X_2(w)\} \leq c\}
       = \{w| X_1(w) \leq c\} \cap \{w| X_1(w) \leq c\} \in \mF$, 
       $\{w| \sup_n X_n(w) \leq c\}
       = \cap_{n} \{w| X_n(w) \leq c\} \in \mF$,
       $\{w| \lim\sup_{n\to \infty} X_n(w) \geq c\}
       = \bigcap_{n=1}^{\infty}\bigcup_{i=n}^{\infty} \{w| X_n(w) \geq c \} \in \mF$ ($\lim\sup_{n\to \infty} X_n(w)= \lim_{n\to \infty}(\sup_{i\geq n}X_i(w))$ is the limit of a decreasing series).
       Therefore, $\max\{X_1, X_2\}, \sup_{n} X_n, \lim\sup_{n\to \infty} X_n$ are random variables.
       
\end{enumerate}

\end{document}

