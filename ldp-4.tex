\documentclass{article}
\usepackage{cmll}
\usepackage{amsmath}
\usepackage{amssymb}
\usepackage[thehwcnt=4]{iidef}
\usepackage{url}
\DeclareMathOperator{\mF}{\mathcal{F}}
\DeclareMathOperator{\Bern}{Bern}
\DeclareMathOperator{\Binom}{Binom}
%\DeclareMathOperator{\Var}{Var}
\usepackage[utf8]{inputenc}
\thecourseinstitute{Tsinghua-Berkeley Shenzhen Institute}
%\title{ldp-1}
%\author{zhaof17 }
%\date{March 2021}
\thecoursename{Large Deviation Theory}
\theterm{Spring 2021}
\begin{document}

\courseheader
\name{Feng Zhao}

\begin{enumerate}
\item We define $\phi(x)=\frac{1}{\sqrt{2\pi}}\exp(-\frac{x^2}{2})$
and $Q(t) = \int_t^{+\infty} \phi(x)dx$. To show
$\frac{t}{1+t^2} \phi(t) \leq Q(t) \leq \frac{\phi(t)}{t}$. We make use of integral by part
and the equality $\phi'(t)=-t\phi(t)$.
Then
\begin{align*}
    Q(t) &= \int_t^{+\infty} \frac{\phi'(x)}{-x}dx\\
    &=\frac{\phi(t)}{t}-\int_t^{+\infty}\frac{\phi(x)}{x^2}dx\leq \frac{\phi(t)}{t} \\
\end{align*}
which gives the upper bound.
For lower bound \cite{elements},
since $(\frac{\phi(u)}{u})'=\frac{u\phi'(u)-\phi(u)}{u^2}=-\frac{u^2\phi(u)+\phi(u)}{u^2}=-(1+\frac{1}{u^2})\phi(u)$
\begin{align*}
    \left(1+\frac1{x^2}\right)Q(t) &=\int_x^\infty \left(1+\frac1{x^2}\right)\phi(u)\,du \\&>\int_x^\infty \left(1+\frac1{u^2}\right)\phi(u)\,du =-\biggl.\frac{\phi(u)}u\biggr|_x^\infty
=\frac{\phi(x)}x.
\end{align*}
\item To show $\sup_{t\geq 0}f(t)=f(0)=\frac{1}{2}$
where $f(t)= Q(t) \exp(t^2/2)$, we take the derivative
$f'(t)=-\phi(t)\exp(t^2/2) + tQ(t)$.
From Problem 1, $Q(t)\leq \phi(t)/ t$, then
$f'(t)\leq -\phi(t)\exp(t^2/2)+\phi(t)\leq 0$ since $\exp(t^2/2)\geq 1$. Therefore, $f(t)$ decreases
in the interval $[0,+\infty)$, and $\sup_{t\geq 0}f(t)=f(0)$.
\item 
\begin{enumerate}
    \item For Poisson distribution,
    $\E[e^{\lambda X}] = \sum_{k=0}^{+\infty}
    e^{k\lambda} \frac{\theta^k e^{-\theta}}{k!}
    =\exp(e^{\lambda }\theta - \theta)$.
    Therefore, the log-MGF $\psi_X(\lambda)= \theta(e^{\lambda} - 1)$. We solve
    $t=\psi_X'(\lambda)$ and get $\lambda = \log(t/\theta)$, therefore, $\psi_X^*(t)=
    \theta - t + t\log\frac{t}{\theta}$
    \item For Bernoulli distribution,
    its log-MGF $\psi_X(\lambda)=\log(pe^{\lambda}+1-p)$.
     We solve
    $t=\psi_X'(\lambda)$ and get $\lambda = \log\frac{t(1-p)}{p(1-t)}$, therefore, $\psi_X^*(t)=
    t\log\frac{t(1-p)}{p(1-t)}-\log\frac{1-p}{1-t}
    =D_{\mathrm{KL}}(\mathrm{Bern}(t)||\mathrm{Bern}(p))$
    \item For Exponential distribution,
    \begin{equation*}
    \E[e^{\lambda X}] = \int_0^{+\infty}
    \theta e^{-\theta x}e^{\lambda x}dx
    =
    \begin{cases}
    \frac{\theta}{\theta - \lambda} & \lambda < \theta \\
    +\infty & \lambda \geq \theta
    \end{cases}
    \end{equation*}
Its log-MGF 
 \begin{equation*}
 \psi_X(\lambda)=
 \begin{cases}
 \log\frac{\theta}{\theta - \lambda} & \lambda < \theta \\
    +\infty & \lambda \geq \theta
  \end{cases}
 \end{equation*}
     We solve
    $t=\psi_X'(\lambda)$ and get
    \begin{equation*}
    \lambda = \begin{cases}\theta - \frac{1}{t} & t > 0
    \\
    -\infty & t \leq 0
    \end{cases}
 \end{equation*} Therefore,
    \begin{equation*}
 \psi^*_X(t)=
 \begin{cases}
+\infty & t \leq 0 \\
    t\theta - 1 -\log(t\theta) & t >0
  \end{cases}
 \end{equation*}
    
\end{enumerate}

\end{enumerate}
\begin{thebibliography}{9}
\bibitem{elements} \url{https://en.wikipedia.org/wiki/Q-function#Bounds_and_approximations}
\end{thebibliography}
\end{document}

